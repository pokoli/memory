\chapter{Introduccio}

El projecte consisteix en implementar una versió multi-jugador del Joc de la Botifarra. Aquest joc no ha de necesitar la instal·lació de cap software especific, sinó que ha de ser accessible a través de qualsevol navegador web modern.

\section{El perquè de tot plegat}

\section{Objectius}
L'objectiu principal d'aquest projecte es la presa de contacte amb noves tecnologies. Al capítol \ref{sec:tecnologies} es detalla més profundament quines tecnologies s'han utilitzat durant el projecte i perquè s'han utilitzat aquestes i no unes altres. 

Per tal de posar en joc totes aquestes tecnologies el JM em va proposar crear un joc en línia. Ell em va proposar crear un domino o un parxís. Degut al meu fanatisme per la botifarra i que feia molts dies que havia pensat en crear un joc així, vaig proposar de realitzar un joc de la botifarra, que és el que finalment s'ha realitzat. 

Personalment vaig posar un objectiu addicional al projecte: Utilitzar el Test Driven Development per tal de poder veure amb primera persona com funciona, i descobrir les ventatges i inconvenients que té usar-lo. 

Com a altres objectius del projecte podem nombrar els següents: 
\begin{itemize}
	\item{Tenir contacte amb altres usuaris de la comunitat internacional que estiguin implicats amb projectes similars.}
	\item{Aprendre a utilitzar forges per a compartir codi (i informació relacionada amb el mateix) amb altres usuaris.}
\end{itemize} 

Aquest objectius venen implícits en el mateix projecte degut a que aquest s'està desenvolupant en el Màster amb Enginyeria del Programari Lliure. 

