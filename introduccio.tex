\chapter{Introducció}

El projecte consisteix en implementar una versió multi-jugador del Joc de la Botifarra. Aquest joc no ha de necessitar la instal·lació de cap software especific, sinó que ha de ser accessible a través de qualsevol navegador web modern.

\section{Objectius}
L'objectiu principal d'aquest projecte és la presa de contacte amb noves tecnologies. Al capítol \ref{sec:tecnologies} es detalla més profundament quines tecnologies s'han utilitzat durant el projecte i perquè s'han utilitzat aquestes i no unes altres. 

Per tal de posar en joc totes aquestes tecnologies el JM em va proposar crear un joc en línia. Ell em va suggerir crear un domino o un parxís. Degut al meu fanatisme per la botifarra i que feia molts dies que havia pensat en crear un joc així, vaig proposar de realitzar un joc de la botifarra, que és el que finalment s'ha realitzat. 

Personalment vaig posar un objectiu addicional al projecte: utilitzar el Test Driven Development per tal de poder veure en primera persona com funciona, i descobrir els aventatges i inconvenients que té usar-lo. 

Com a objectius addicionals del projecte podem esmentar els següents: 
\begin{itemize}
	\item{Tenir contacte amb altres usuaris de la comunitat internacional que estiguin implicats en projectes similars.}
	\item{Aprendre a utilitzar forges per a compartir codi (i informació relacionada amb el mateix) amb altres usuaris.}
\end{itemize} 

Aquest objectius venen implícits en el mateix projecte degut a que aquest s'està desenvolupant en el Màster amb Enginyeria del Programari Lliure. 

\section{Estructura del document}

Aquest document s'estructura de la següent forma: 

Al Capítol \ref{chap:botifarra} s'introdueix el joc de la botifarra, les seves regles i l'estructura d'una partida. L'objectiu d'aquest capítol és proporcionar a les persones que no saben jugar aquest jocs els coneixements bàsics per poder disfrutar d'aquest joc. 

Al Capítol \ref{chap:codi_obert} s'explica quines són les implicacions de realitzar un projecte de codi obert. També es valora, a grans trets, quins són els beneficis i els inconvenients de realitzar un projecte de codi obert.

Al Capítol \ref{chap:metodologia} s'explica quines metodologies de desenvolupament s'han utilitzat durant el transcurs del projecte.

Al Capítol \ref{sec:tecnologies}, s'explica quines són les tecnologies que s'han utilitzat per al desenvolupament del projecte.

Al Capítol \ref{sec:Node.js-full}, s'explica que és el framework Node.js, quines són les seves implicacions i perquè s'ha utilitzat per al projecte. 

Al Capítol \ref{chap:requeriments} s'explica quines són les funcionalitats del sistema que es vol implementar. També s'explica l'estructura bàsica del mateix. 

Al Capítol \ref{chap:dependencies} s'explica quines són les dependències del projecte i les tasques que s'han de realitzar per a la seva instal·lació. 

Al Capítol \ref{chap:implementacio} es poden trobar els detalls de les tasques realitzades per la implementació del projecte. 

Al Capítol \ref{chap:interficie_usuari} s'exposen les consideracions que s'han dut a terme per al disseny de la interfície d'usuari.

Al Capítol \ref{chap:desplegament} es detalla quins son els requeriments per desplegar el projecte, quines opcions existeixen disponibles i quines són les tasques necessàries per al desplegament del mateix. 

Finalment, al Capítol \ref{chap:conclusions} s'expliquen les conclusions a les que s'han arribat gràcies al desenvolupament d'aquest projecte. També s'expliquen linies d'ampliació del mateix.
