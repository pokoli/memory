\chapter{Tecnologies}
\label{sec:tecnologies}

El programa que es vol desenvolupar durant el projecte ha d'utilitzar les següents tecnologies: 

\begin{enumerate}
	\item{Test Driven Development com a metodologia de programació.}
	\item{Javascript com a llenguatge de programació.}
	\item{Node.js com a entorn d'execució del servidor.}
	\item{Websockets per a la comunicació entre el client i el servidor.}
	\item{HTML5,CSS i JQuery per a la capa de visualització del client.}
	\item{git com a programa de control de versions, utilitzant github per publicar el codi}
\end{enumerate}

\section{Perquè aquestes tecnologies?}

Abans de començar el projecte, quan en JM em va proposar de realitzar-lo em va dir que l'objectiu principal d'aquest projecte era provar, i a la vegada investigar el seu funcionament , noves tecnologies. Les que em va proposar són les següents: 
\begin{itemize}
	\item{Node.js}
	\item{Websockets}
	\item{HTML5, especialment l'objecte canvas}
\end{itemize}

Per la meva part desconeixia totalment l'existència de node.js i dels Websockets. Tot i tenia petits coneixements sobre HTML5, sabia que aquest no eren molt extensos i que eren necessari expandir-los.

El fet d'explorar noves tecnologies va ser molt motivador per mi, fins al punt de dir que ha estat una de les principals raons per decidir-m'hi per aquest projecte i no per un altre. 

Per la meva part vaig decidir utilitzar Test Driven Development com a metodologia de programació perquè havia llegit alguna cosa sobre ell i em semblava bastant interessant. Tot i que normalment utilitzo el disseny iteratiu, aquest treball em va semblar l'excusa perfecta per poder provar nous tipus de desenvolupament i així poder valorar quin d'ells és millor. 

Després d'explorar node.js vaig veure que el seu codi esta allotjat a github, i que la majoria dels mòduls que corren sobre ell també. Així em va semblar convenient allotjar tot el codi del projecte al mateix lloc. Github utilitza git com software de control de versions

\section{Test Driven Development}

Test Driven Development o desenvolupament de software bastat en Testos es una metodologia de desenvolupament de software centrada en els jocs de proves. Aquesta metodologia utilitza el següent cicle de desenvolupament: 

\begin{figure}[htbp]
\centering\includegraphics{test-driven-development.png}
\caption{Representació del cicle de desenvolupament basat amb TDD}
\label{fig:tdd}
\end{figure} 

\begin{enumerate}
    \item{Escriure un joc de prova.}
    \item{Executar tots els jocs de prova per comprovar que el nou test falla.}
    \item{Escriure el codi necessari per a que el joc de prova sigui favorable.}
    \item{Executar tots els jocs de provar per comprovar que tots els test son satisfactoris.}
    \item{Reestructurar el codi}
\end{enumerate}

A la figura \ref{fig:tdd} es pot veure un diagrama del cicle de desenvolupament basat en Test Driven Development.

Els passos descrits anteriorment es duen a terme per a cada nova funcionalitat que es vol implementar. Aquesta metodologia té l'objectiu d'implementar petites funcionalitats que puguin ser provades de forma independent per un joc de prova, per tal de desprès ajuntar-les amb el conjunt de l'aplicació. 

%TODO: Fer una taula aqui?????
El Test Driven Development té les següents ventatges: 

\begin{enumerate}
    \item{Solidesa davant les fallades.}
    \item{Claredat dels requisits a implementar.}
    \item{Millora de la productivitat.}
\end{enumerate}

i els següents inconvenients: 

\begin{enumerate}
    \item{Difícil d'implementar en alguns casos: Interfícies de client,bases de dades, programes dependents de la xarxa }
    \item{Dependència de que els testos estiguin ben escrits. }
    
\end{enumerate}



\section{Javascript}

Javascript és un llenguatge script basat en el concepte de prototipus (herència per delegació), implementat originàriament per Netscape Communications Corporation l'any 1995, i que va derivar en L'estàndard ECMAScript. És conegut sobretot pel seu ús en pàgines web, però també s'utilitza en altres aplicacions.

En un principi, s'usava a pàgines web HTML, per realitzar feines i operacions al marc de l'aplicació client. Amb l'aparició de la Web 2.0, Javascript s'ha convertit en un veritable llenguatge de programació que aporta la potència de càlcul al navegador per augmentar l'usabilitat d'aplicacions web amb tècniques avançades com AJAX o JCC.

%TODO: (veure si cal) Explicar que es el DOM

\section{Node.js}

Node.js es un framework dissenyat per escriure aplicacions web altament escalables. Els programes s'escriuen amb Javascript. La seva estructura es està basada en events i una entrada/sortida asíncrona. Aquest framework utilitza el motor de Javascript V8\footnote{\url{http://en.wikipedia.org/wiki/V8_JavaScript_engine}}. El motor V8 és de codi obert i està desenvolupat per Google, ja que aquest l'utilitza al seu navegador Google Chrome. Aquest motor millora el rendiment compilant el codi Javascript a codi natiu abans d'executar-lo.

%TODO: Explicar més profundament que vol dir bastat en events i entrada/sortida asincrona

%Tot i que node.js es un framwork bastant inusual, les primeres impresions ell van ser magnífiques. 
%Primer de tot utilitza codi Javascript per a crear servidors, quan tots els meus coneixements eren per executar codi Javascript en l'entorn de client. La segona cosa que em va sorprendre molt de node.js va ser que aquest sigui un entorn d'execució de programés asíncrons. 

\section{Websockets}

WebSocket és una tecnologia que proporciona un canal de comunicació bidireccional i full-duplex sobre un únic socket TCP. Està dissenyada per ser utilitzada en navegadors i servidors web, però pot utilitzar-se per qualsevol aplicació client/servidor. La API de WebSocket s'està normalitzant pel W3C\footnote{World Wide Web Consortium}, i el protocol WebSocket, a la vegada, s'està normalitzant per l'IETF\footnote{Internet Engineering Task Force}. Normalment, per temes de seguretat, els administradors de xarxes bloquegen les connexions als ports diferents del port 80, així els WebSockets pretenen solucionar aquest problema, aportant una tecnologia que proporcioni una funcionalitat similar a la que s'obté obrint diferents connexions en diferents ports però multiplexant els diferents serveis de a través d'un únic port TCP.

\section{HTML5}

HTML representa les sigles de \emph{Hyper Text Markup Language}, que en català signifiquen "llenguatge de marcat d'hipertext". És un llenguatge de marcat que deriva de l'SGML dissenyat per estructurar textos i relacionar-los en forma d'hipertext. Gràcies a Internet i als navegadors web, s'ha convertit en un dels formats més populars que existeixen per a la construcció de documents per a la web. 

HTML5 és la última versió de HTML. Aquesta última versió proporciona noves marques per poder estructurar els documents d'una forma més semàntica. A més a més proporciona noves etiquetes per poder etiquetar contingut multimèdia, ja sigui àudio o vídeo. També afegeix una nova etiqueta "canvas" que permet la representació dinàmica de formes 2D i imatges a través de llenguatge Javascript. 

\section{CSS}

CSS representa les sigles de \emph{Cascading Style Sheets}, que en catalpa signifiquen "Fulls d'estil en cascada". és un llenguatge de fulls d'estil utilitzat per descriure la semàntica de presentació (l'aspecte i format) d'un document escrit en un llenguatge de marques. La seva aplicació més comuna és dissenyar pàgines web escrites en HTML i XHTML, però el llenguatge també pot ser aplicat a qualsevol classe de document XML, incloent-hi SVG i XUL.

El principal objectiu de CSS es permetre la separació del contingut d'un document escrit amb un llenguatge de marques de la seva presentació. Aquesta separació permet que múltiple pàgines comparteixin un format comú i també que un mateix document de marcatge es pugui mostrar en diferents formats. 

CSS defineix regles d'estil per a cada element, on s'especifica com es mostrarà cada element. Així també definies un esquema de prioritat per determinar quines regles d'estil s'apliquen si més d'una regla esta associada amb un element en particular. Aquestes característiques han permès que avui en dia sigui el llenguatge més utilitzat a l'hora de definir l'aspecte i el format d'una pàgina web. 

\section{JQuery}

JQuery és una llibreria de Javascript, creada inicial ment per John Resig, que permet simplificar la forma d'interactuar amb els documents HTML, manipular l'arbre DOM, interactuar amb els events, desenvolupar animacions y afegir interacció dinàmica a pàgines web. 

JQuery es software de codi obert, llicenciat sota les llicències MIT y LGPLv2, permeten ser usada tan a projectes lliures com a projectes privatius. 

\section{Git i Github}

Git es un sistema de control de versions \footnote{\url{http://es.wikipedia.org/wiki/Control_de_versiones}} dissenyat per Linus Trovalds, creador de Linux. Git té les següents característiques principals: 

\begin{description}
    \item[Suport per desenvolupament no lineal] Git està dissenyat per tal de proporcionar una interfície ràpida per a crear branques i mesclar-les.
    \item[Distribuït] Cada usuari té un repositori propi. Aquests respositoris poden ser fusionats amb respositoris d'altres usuaris.
    \item[Compatibilitat amb protocols existents] Els respositoris poden ser publicats mitjançant protocols HTTP,FTP,Rsync i fins i tot un protocol propi.
    \item[Eficient per grans projectes] Git es ràpid i escalable. A més a més no es fa més lent a mesura que la història del projecte es fa gran, com passa amb altres sistemes.
    \item[Portable] Funciona tan en sistemes Linux,Unix,MAC i Windows.
\end{description}

Github un servei de d'allotjament de codi per a repositoris Git. A part d'allotjar el teu propi repositori ofereix altres serveis com poden ser Gist\footnote{\url{https://gist.github.com/}} (per a compartir petits trossos de codi), una wiki per projecte, allotjament de pàgines web i un petit gestió d'incidències molt flexible. A data 16/12/2011 compta amb 1,184,001 persones i allotja 3,448,813 repositoris.

Github ofereix il·limitats repositoris de forma gratuïta sempre que el seu codi sigui públic. A més a més ofereix repositoris privats per un petit preu mensual. 


