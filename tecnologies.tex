\section{Tecnologies}
\label{sec:tecnologies}

El programa que es vol desenvolupar durant el projecte ha d'utilitza les següents tecnologies: 

\begin{enumerate}
	\item{Test Driven Development com a metodologia de programació.}
	\item{Javascript com a llenguatge de programació.}
	\item{Node.js com a entorn d'execució del servidor. }
	\item{Websockets per a la comunicació entre el client i el servidor.}
	\item{HTML5,CSS i JQuery per a la capa de visualització del client.}
	\item{git com a programa de control de versions, utilitzant github per publicar el codi}
\end{enumerate}

\subsection{Perquè aquestes tecnologies?}

Abans de començar el projecte, quan en JM em va proposar de realitzar-lo em va dir que l'objectiu principal d'aquest projecte era provar, i a la vegada investigar el seu funcionament , noves tecnologies. Les que em va proposar són les següents: 
\begin{itemize}
	\item{Node.js}
	\item{Websockets}
	\item{HTML5, especialment l'objecte canvas}
\end{itemize}

Per la meva part desconexeia totalment l'existència de node.js i dels websockets. Tot i tenia petits coneixements sobre HTML5, sabia que aquest no eren molt extensos i que eren necessàri expandir-los.

El fet d'explorar noves tecnologies va ser molt motivador per mí, fins al punt de dir que ha estat una de les principals raons per decidir-m'he per aquest projecte i no per un altre. 

Per la meva part vaig decidir utilizar Test Driven Development com a metodologia de progamació perquè havia llegit alguna cosa sobre ell i em semblava bastant interesant. Tot i que normalment utilizo el diseny iteratiu, aquest treball em va semblar l'escusa perfecta per poder provar nous tipus de desenvolupament i així poder valorar quin d'ells és millor. 

Després d'explorar node.js vaig veure que el seu codi esta allotjat a github, i que la majoria dels móduls que corren sobre ell també. Així em va semblar convenient allotjar tot el codi del projecte al mateix lloc. Github utiliza git com software de control de versions

\subsection{Test Driven Development}

\subsection{Javascript}

\subsection{Node.js}
%Tot i que node.js es un framwork bastant inusual, les primeres impresions ell van ser magnífiques. 
%Primer de tot utilitza codi Javascript per a crear servidors, quan tots els meus coneixements eren per executar codi Javascript en l'entorn de client. La segona cosa que em va sorprendre molt de node.js va ser que aquest sigui un entorn d'execució de programés asíncrons. 

\subsection{Websockets}

\subsection{HTML5}

\subsection{CSS}

\subsection{JQuery}

\subsection{Git i GitHub}


