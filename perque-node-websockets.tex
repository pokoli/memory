\chapter{Perquè Node.js}
\label{sec:node.js-full}

Node.js és una entorn d'execució per a codi Javascript, és a dir es capaç d'interpretar codi escrit amb Javascript. Tradicionalment Javascript s'ha utilitzat per executar codi dins del navegador, el que s'anomena la banda del client. Node.js ha modificat totalment aquest concepte i ens permet executar codi Javascript en un servidor. Així Node.js ens proporciona els mètodes suficients per poder interactuar amb el sistema que l'esta executant.

Node es va dissenyar amb els següents objectius: 

\begin{itemize}
    \item{Evitar les operacions d'Entrada/Sortida bloquejants.}
    \item{No emmagatzemar res amb buffers, enviar-ho directament al destinatari.}
    \item{Tenir suport integrat pels protocols més utilitzats: TCP, DNS, HTTP.}
    \item{Poder executar-se en múltiples plataformes, per no crear una dependència.}
\end{itemize}

\section{Perquè Javascript?}

Javascript té les següents característiques: 

\begin{enumerate}
    \item{Dinàmic.}
    \item{De tipatje feble.}
    \item{Orientat a Objectes.}
    \item{Suporta funcions anònimes.}
    \item{Està dissenyat per suportar esdeveniments.}
\end{enumerate}

Aquestes característiques el fan el candidat ideal per a proporcionar una infraestructura sense bloquejos, basada en esdeveniments. A més a més, és un llenguatge que està molt extes en el món de la informàtica, cosa que provoca que hi hagi una gran quantitat de desenvolupadors que en tinguin coneixements. El principal ús de Javascript és el processament d'esdeveniments a la web. Així la semàntica de Node.js s'ha dissenyat de forma que sigui semblant al processament d'esdeveniments a la web, per tal facilitat la corba d'aprenentatge dels programadors amb coneixements de desenvolupament web.

\section{Codi Síncron vs. Codi Asíncron}
\label{sec:sincron-ascinron}

Com s'ha comentat a la secció \ref{sec:node.js-min}, Node.js permet executar codi de forma asíncrona, proporcionant en un entorn d'execució d'un sol fil d'execució al desenvolupador, encara que internament n'utilitzi més d'un. Així en aquesta secció el que farem es explicar les diferències entre el codi síncron i asíncron.

Un dels primers programes que aprèn qualsevol programador és el següent:

\begin{lstlisting}
//Output the string
puts("Please Enter Your Green Name");
//wait for and take input
var green_name=gets();
//output the string
puts("Your Green Name is: "+green_name);
\end{lstlisting}

Aquest programa pregunta a l'usuari quin és el seu nom i l'escriu per pantalla. Així mentre l'usuari no introdueix cap text el programa es queda a l'espera i no pot executar cap més tipus de codi. Aquest el procediment generar de qualsevol proces síncron: no realitzar res mentre s'espera alguna informació externa.

En les següents línies de codi podeu observar el mateix codi implementat de forma asíncrona:
	
\begin{lstlisting}
//Output the string
puts("Please Enter Your Green Name");
//declare a call back function
gets( function (green_name) {
   puts("Your Green Name is: "+green_name);
});
//do not wait, do something else
\end{lstlisting}

La gran diferència entre aquest codi i l'anterior es troba en la línia 4. En aquesta cas el que fem es especificar el codi que volem que s'executi després d'obtenir la resposta de l'usuari, i continuar amb l'execució del proces, sense esperar la resposta de l'usuari per poder realitzar altres tipus d'accions. 

El fet de que el codi s'executi de forma asíncrona, implica que el programa pot realitzar diferents tasques mentre espera que el sistema realitzi alguna altra tasca. Això vol dir que el programa pot seguir realitzant tasques mentre espera que es realitzi una consulta a la base de dades, que es llegeixi un fitxer des de disc, o que es rebin les dades d'un servidor extern. Així el programa s'estructura amb el que es coneix com a callbacks. Un callback es el codi que s'executara després d'haver realitzat alguna tasca. 

\begin{lstlisting}
fs.readFile(__dirname + '/someDir/someFile', 'utf8', function (err,data) {
  if (err) {
    return console.log(err);
  }
  console.log(data);
});
\end{lstlisting}

En aquest text es pot veure el codi necessari per a llegir un fitxer de disc. Així el que es fa es dir-li al sistema, llegeix el fitxer /someDir/someFile, i en el moment que tinguins alguna informació executa la funció que es passa a l'ultim paràmetre. Amb això aconseguim que mentre el proces estigui llegint el fitxer nosaltres puguem seguir fent càlculs, i que quan s'hagi acabat la lectura del fitxer puguem fer amb ell el que creguem convenient. 

El processament asíncron és fa gracies als anomenats esdeveniments, que s'encarreguen d'executar codi font quan s'ha produït alguna cosa. A més a més, Node.js ens permet crear esdeveniments personalitzats i fer que s'executi el nostre codi cada vegada que es produeix un esdeveniments. Els esdeveniments han estat utilitzats en gran part del desenvolupament del projecte. 

\section{El bucle d'esdeveniments}

Actualment s'utilitzen les següents tècniques de programació: 

\begin{enumerate}
    \item{Codi síncron en un sol fil d'execució.}
    \item{Codi síncron en un més d'un fil d'execució.}
    \item{Codi asíncron en diferents fils d'execució.}
\end{enumerate}

En el primer cas la CPU només es capaç d'executar el codi d'un programa a la vegada i aquesta roman inactiva mentre el programa realitza altres tasques (llegir un fitxer, esperar l'entrada d'informació per part de l'usuari,etc.). La programació d'aquest tipus de programes es molt simple ja que no et cal preocupar de si pot haver altres fils d'execució llegint les mateixes variables que el fil actual, encara que el seu rendiment es baix en entorns d'alta concurrència. 

En el segon cas la CPU s'utilitza per diversos fils d'execució a la vegada, així mentre un fil d'execució està ocupat llegint un fitxer la CPU pot ser aprofitada per una altre fil d'execució. De totes formes aquesta forma de programació pot arribar a ser molt confusa en el cas que més d'un fil d'execució hagi de modificar la mateixa informació. Aquests factors s'han de tenir en compte per part del desenvolupador a l'hora de desenvolupar el programa. 

L'ultim cas és el que s'utilitza a Node.js, amb la diferència que Node.js gestiona internament els diferents fil d'execució, evitant que el desenvolupador s'haigi de preocupar dels mateixos. Per aconseguir això el que fa és posar referències a funcions en una cua d'execució, així cada vegada que la CPU quedi lliure s'executa una nova porció de codi, evitant que més d'una porció de codi s'executi de forma simultània i evitant que la CPU quedi a l'espera de que es produeixi un esdeveniments. Tota la gestió referent als esdeveniments és el que s'anomena el bucle d'esdeveniments. 

En Node.js hi ha dos formes d'afegir l'execució d'un tros de codi al bucle d'esdeveniments: 

\begin{enumerate}
    \item{Sense associar-ho a cap esdeveniment, el codi es posa a la cua i s'executa la pròxima vegada que la cua estigui lliure.}
    \item{Associant el codi a l'ocurrència d'un esdeveniment. Aquest codi es posarà a la cua d'execució cada vegada que es produeixi l'esdeveniment associat.}
\end{enumerate}

La primera es realitza a través de la funció \emph{process.nextTick(callback)}, que s'encarrega d'afegir a la cua d'execució la funció que s'ha passat a la variable callback. En les següents línies de codi es pot veure un exemple d'aquest procediment. En aquest exemple s'afegeix una funció que escriurà per pantalla quan el bucle d'esdeveniments determini que toca la seva execució.

\begin{lstlisting}
process.nextTick(function () {
  console.log('nextTick callback');
});
\end{lstlisting}

La segona és una mica més complicada, i la seva implementació és realitza a través de la funció constructora EventEmmiter. En el prototip d'aquesta funció s'implementen els següents atributs: 

\begin{description}
    \item[addListener(event, listener)]{ Associa la funció passada a la variable \emph{listener} per que s'executi cada vegada que es produeix un event del tipus \emph{event}}
    \item[on(event, listener)] { Exactament el mateix funcionament que la funció \emph{addListener}.}
    \item[once(event, listener)] {Té el mateix comportament que la funció \emph{addListener}, excepte que aquesta funció només s'executa la primera vegada que es produeix l'esdeveniment.}
    \item[removeListener(event, listener)]{ Desassocia la funció \emph{listener} dels esdeveniments del tipus \emph{event}.}
    \item[removeAllListeners(event)] {Elimina totes les funcions associades als esdeveniments de tipus \emph{event}. Si no s'especifica cap tipus d'esdeveniment s'eliminen totes les funcions associades al objecte.}
    \item[listeners(event)] { Retorna una llista de funcions associades a l'esdeveniment de tipus \emph{event}.}
    \item[setMaxListeners(n)] {Limita el nombre de funcions associades a cada tipus d'esdeveniment a \emph{n}.}
    \item[emit(event, arguments)] { Executa totes les funcions associades a l'esdeveniment de tipus \emph{event} passant \emph{arguments} com a paràmetres de totes les funcions.}
\end{description}

Aquest conjunt de mecanismes permet que nosaltres puguem crear els nostres propis esdeveniments, i que puguem associar l'execució de determinat codi cada vegada que es produeix un d'ells. Per tal de fer això simplement hem de crear una subclasse de event.EventEmmiter i que executi les funcions associades a cada event mitjançant la funció emit. 

A les següents línies de codi es pot observar un petit exemple de com funciona a gestió d'esdeveniments. En aquestes línies s'ha suposat que la variable server està basada en la funció constructora event.EventEmitter. A més a més també s'ha suposat que es produeix un event del tipus \emph{connection} cada vegada que un client es connecta al nostre servidor. 

\begin{lstlisting}
server.on('new-stream',function(stream){
    console.log(stream);
});

server.on('stream-error',function(msj){
    console.log('An error ocurred with the stream: '+msj);
});

server.on('connection', function (stream) {
  if(stream) 
    this.emit('new-stream',strem);
  else
    this.emit('stream-error','Unable to determine the stream' );
});
\end{lstlisting}

El seu funcionament és el següent: 

\begin{description}
    \item[línies 1-3] {S'associa una funció a l'esdeveniment \emph{new-stream}. Aquest funció el que fa es escriure les dades que es passen a la variable stream per la pantalla.}
    \item[línies 5-7] {S'associa una funció a l'esdeveniment \emph{stream-error}. Aquest funció el que fa es escriure el missatge d'error que es passa a la variable \emph{msj} per la pantalla.}
  \item[línies 9-13] {S'associa una nova funció a l'esdeveniment \emph{connection} (que s'executa cada vegada que un client es connecta al servidor); aquesta funció el que fa es comprovar si els client ens ha passat dades (stream), i executa un dels dos esdeveniments comentats anteriorment.}
\end{description}

\section{Beneficis de Node.js}

Node no té bloquejos, no te diferents fils d'execució competint pel mateix recurs. Node.js només espera a que es produeixi algun esdeveniment i quan es produeix el processa. Això es transforma en codi molt ràpid. Tota aquesta lògica també es podría programar amb altres llenguatges implementant un servidor multifil, encara que sería molt més complicat de programar-ho ja que ens hauríem de preocupar dels problemes relacionats amb la concurrència. 

El disseny d'un joc en línia requereix que no hi hagi bloquejos, ni diferents fils d'execució competint pel mateix recurs, i que el codi s'executi el més ràpid possible. Tots aquest fets provoquen que Node.js i el seu disseny basat en esdeveniments sigui la millor elecció per allotjar tota la lògica d'un lloc en línia i per poder suportar el major nombre de partides de forma simultània.


