\chapter{La botifarra}
\label{chap:botifarra}

En aquesta secció es descriurà com funciona el Joc de la botifarra i les seves regles. A part també es descriuran els aspectes a tenir en compte de cara al programa. 


El joc es juga entre quatre jugadors, que formant parella, seuran l'un al davant de l'altre. S'utilitza una baralla de 48 cartes formada per quatre colls (oros, copes, espases i bastos) amb 12 cartes cadascun, altrament coneguda com la baralla espanyola.


L'objectiu de la partida és superar els 100 punts abans que la parella contrària. 


\section{Puntuació}

El valor de cadascuna de les cartes és:
\begin{itemize}
	\item{Manilla (9), 5 punts}
    \item{As (1), 4 punts}
    \item{Rei (12), 3 punts}
    \item{Cavall (11), 2 punts}
    \item{Sota (10), 1 punt}
\end{itemize}

La resta de cartes no tenen cap punt.
    
Dins d'un mateix coll, l'ordre de les cartes de més alta a més baixa és el següent: 9, 1, 12, 11, 10, 8, 7, 6, 5, 4, 3 i 2.

Per cada basa\footnote{Una basa esta formada per 4 cartes que ja s'han jugat.} guanyada es compta un punt. Això vol dir que en cada mà o dat hi ha 72 punts a repartir entre les dues parelles.

La parella guanyadora de cada mà, s'apuntarà la quantitat de punts que passi de 36. Cal tenir en compte que aquesta quantitat s'haurà de multiplicar per dos, quatre o vuit si hi ha hagut un contro, re-contro o Sant Vicenç. Aquest resultat s'haurà de doblar si el trumfo és botifarra.

\section{Mecànica de joc}

El joc de la botifarra es composa per un nombre variables de rondes. De cada ronda es computa la puntuació tal hi com s'ha comentat a l'apartat anterior. Així, es van jugant rondes fins que un dels dos equips es proclama guanyador.

\subsection{Preparació de la partida}

Abans de començar a jugar una partida, cal establir quin jugador ha de ser el primer en repartir les cartes i triar triomf o trumfo. Per això un dels quatre jugadors atribueix un pal a cada jugador i aleatòriament gira una carta de la baralla. Aquell jugador que tenia atribuït el pal de la carta girada, serà el primer en repartir les cartes.

El jugador situat a l'esquerra del que triarà el trumfo barrejarà les cartes. Quan ho hagi fet les deixarà a la taula de cara avall perquè el seu company les talli o escapci.

\subsection{Transcurs de la partida}

El jugador que de triar el trumfo agafarà i les cartes i les repartirà sense mirar-les ni mostrar-les als altres jugadors, de quatre en quatre successivament a cada jugador començant pel que té situat a la seva dreta fins a repartir les 48 cartes. En conseqüència, en total donarà tres voltes senceres en sentit antihorari, repartirà en total 12 cartes a cada jugador i el jugador que reparteix es quedarà les últimes quatre cartes.

Un cop repartides les cartes i abans de començar a jugar, cal escollir trumfo. El jugador a qui correspongui escollir trumfo (el mateix que ha repartit), i a la vista de les cartes que té a la mà, haurà d'escollir ell el trumfo, o delegar al company.

Per tal d'indicar el trumfo que s'ha escollit, cal dir en veu alta el pal corresponent: oros, copes, espases o bastos. El pal que s'hagi escollit és el que mana en la mà. També hi ha l'opció de dir "botifarra". En aquest cas no hi ha cap pal que mani sobre els altres. També es pot passar el torn al company, aquest es veu obligat a fer trumfo escollint una de les opcions anteriors. (Oros, Copes, Espases, Bastos o Botifarra). Si s'escull botifarra, la quantitat de punts que s'apuntarà la parella guanyadora de la mà serà doble.

Un cop s'ha escollit trumfo la parella contrària pot contrar. Contrar, vol dir que la parella que guanyi aquesta mà, sumarà el doble de punts, o el quàdruple si s'ha fet botifarra.Per contrar, s'ha de dir clarament: "contro". Si no es vol contrar, es donarà un cop amb la mà sobre la taula.Pot contrar qualsevol dels jugadors de la parella que no ha fet trumfo. Només es pot contrar un cop a cada mà, és a dir, si un jugador ha contrat, el seu company ja no ho podrà fer.

Un cop la mà està contrada, els components de la parella a qui ha tocat repartir i fer trumfo, poden re-contrar. Re-contrar vol dir que la parella que guanyi sumarà els punts de la mà multiplicats per 4, o per 8 si és botifarra. Per re-contrar, s'ha de dir clarament: "re-contro". Si no es vol re-contrar, es donarà un cop amb la mà sobre la taula.


Un cop escollit el trumfo i contrat, re contrat o Sant Vicenç, ja podem començar a jugar les cartes. El primer jugador a tirar una carta, és el que està a la dreta de qui ha repartit, per a això, n'escollirà una de les que té a la mà i la deixarà davant seu sobre la taula cara enlaire.

A continuació, la resta de jugadors aniran jugant per ordre contrari al de les busques del rellotge, fins a completar la basa. La parella que hagi guanyat la basa, recollirà les cartes i les guardarà cara avall en un pilot. El jugador que guanyi la basa començarà la següent, i així fins a jugar totes les cartes.

Les cartes ja jugades i dipositades al pilot de cada parella, no es podran consultar, a excepció de l'última basa afegida a cada pilot.

Un cop jugades totes les cartes es comptaran els punts que ha fet cada parella, la parella que hagi guanyat la mà s'anotarà els punts corresponents.

En aquest punt s'haurà jugat la primera ronda. Per tal de continuar amb la següent ronda el jugador que ha repartit i escollit trumfo a la mà anterior serà l'encarregat de barrejar les cartes, el seu company de tallar la baralla, i el de la seva dreta de repartir i escollir trumfo, i així successivament fins al final de la partida.

La partida s'acaba quan una de les parelles ha aconseguit superar els 100 punts.

\section{Normes per jugar les cartes.}

Per saber quines cartes es poden jugar, s'han de seguir les normes següents:

Pel primer jugador a jugar la basa:

\begin{itemize}
    \item{Pot jugar qualsevol carta. El pal d'aquesta carta és el que anomenarem pal de sortida.}
\end{itemize}

Per la resta de jugadors les normes són les següents:
\begin{enumerate}
\item{Si la basa va del company (la carta del company és la que està guanyant la basa)
    \begin{itemize}
        \item{Si el jugador té cartes del pal de sortida haurà de jugar obligatòriament una d'aquestes sense necessitat de superar-la.}
        \item{Si el jugador no té cartes del pal de sortida pot jugar qualsevol carta.}
     \end{itemize}
}   
\item{Si la basa no va del company (la carta guanyadora és d'un dels jugadors de la parella contrària):
    \begin{itemize}
        \item{Si el jugador té cartes del pal de sortida haurà de jugar-ne una d'aquest pal i sempre que pugui haurà de matar-la.}
        \item{Si el jugador no té cap carta del pal de sortida però en té d'altres que guanyen la basa (triomf) haurà de jugar una carta que guanyi la basa}
        \item{Si el jugador no té ni cap carta del pal de sortida ni cap carta que guanyi la basa, podrà jugar la carta que vulgui.}
    \end{itemize}
}
\end{enumerate}
Com a resum podem dir que únicament estem obligats a matar les cartes dels contraris, no les del company i que sempre (si en tenim) hem de tirar cartes del pal de sortida.

