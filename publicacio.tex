\chapter{Desplegament del programa}
\label{chap:desplegament}

El desplegament del programa consisteix en posar a disposició del públic Wel programa que s'ha desenvolupat. En el nostre cas, com es tracta d'una aplicació basada en web, la tasca consisteix en posar un servidor web que s'encarregui de que la nostra aplicació sigui accessible a través d'una direcció web. 

Ja que el framework Node.js és bastant modern, no existeixen molts proveïdors per a l'allotjament d'aplicacions desenvolupades amb aquest framework. Només s'han trobat dues empreses que es dediquen a l'allotjament de aplicacions desenvolupades amb Node.js i que a més a més ofereixen el desplegament gratuït en el cas de que no es necessitin molts requeriments. Aquestes empreses són NodeJitsu\footnote{\url{http://nodejitsu.com}} i Heroku\footnote{\url{http://www.heroku.com}}

Una altra opció per al desplegament del programa es la utilització d'un Servidor Privat Virtual\footnote{\url{http://es.wikipedia.org/wiki/Servidor_virtual_privado}}. 

Així les opcions disponibles per a la publicació del programa són les següents: 

\begin{itemize}
\item{NodeJitsu}
\item{Heroku}
\item{Servidor virtual propi}
\end{itemize}

%TODO: Explicar les diferents opcions de desplegament del programa.
%Fer una taula comparativa amb les diferents opcions de cadascuna????

De les tres opcions disponibles s'ha elegit Heroku, ja que permet l'allotjament d'una aplicació de forma gratuïta i permeten fer el desplegament a través de git\footnote{Podeu trobar més informació a \url{http://www.koolpi.com/heroku-publicant-aplicacions-web-al-nuvol/} }. Gràcies a que el desplegament es fa a través de git sempre es té constància de quina és la versió actual que hi publicada a Heroku i es pot realitzar una còpia idèntica del codi que es té desplegat, ja que aquest està controlat per un sistema de control de versions.

\section{Publicant a Heroku}

La publicació de la nostra aplicació s'ha dut a terme seguint la documentació de la pàgina oficial de Heroku\footnote{\url{https://devcenter.heroku.com/articles/nodejs}} i el conjunt d'aplicacions toolbelt\footnote{\url{https://toolbelt.herokuapp.com/}}. 

Heroku s'encarrega de gestionar les dependències a través de npm i els fitxers package.json\footnote{Veure la secció \ref{sec:gestio-dependencies}}. Així s'encarrega d'instal·lar de forma automàtica totes les dependències que nosaltres especifiquem en aquest fitxer. Cada vegada que publiquem una nova versió de l'aplicació també s'encarregarà de comprovar si les dependències han canviat i ens instal·larà les noves llibreries en cas de que sigui necessari. 

Les següents adaptacions han es necessàries per poder treballar amb Heroku: 

\begin{itemize}
\item{Modificar l'estructura de directoris eliminant el directori src/.}
\item{Crear un fitxer (Procfile per indicar la comanda que ha d'utilitzar Heroku per executar el servidor.}
\item{Utilitzar el port del medi en cas de que aquest estigui definit. }
\item{Fer que el programa sigui independent de la URL des d'on s'executa.}
\end{itemize}

Una vegada s'ha realitzat aquestes adaptacions i s'ha lligat el nostre repositori local amb el de Heroku es poden pujar els canvis a Heroku mitjançant la comanda: 

\begin{lstlisting}[language=bash]
git push heroku
\end{lstlisting}

A partir d'aquest moment la nostre aplicació serà accessible a través dels servidors de Heroku.

\subsection{Problemes amb heroku}

Desprès d'uns dies de funcionament amb heroku em vaig adonar que la comunicació de WebSockets es tallava. Desprès de parlar amb el suport de heroku ens van comentar que la seva infraestructura no suportava WebSockets. Així es va decidir desplegar la aplicació en un servidor privat virtual per tal de que aquesta funcioni amb totes les seves funcionalitats. 

\section{Publicant a un Servidor propi.}

Per tal de que la nostra aplicació funcioni de forma correcta necessitem que els sistema disposi de les següents utilitats:

\begin{itemize}
\item{Node.js}
\item{Git}
\end{itemize}

Per la instal·lació de Node.js s'han seguit els passos comentats a l'apartat \label{sec:instalacio-Node.js}. Així una vegada s'han instal·lat aquest requeriments podem instal·lar la nostra aplicació seguint els següents passos: 

\begin{enumerate}
\item{Obtenir el codi font del projecte utilitzant git}
\item{Obtenir les dependències del programa, utilitzant npm com s'ha comentat al capítol \ref{sec:gestio-dependencies}}
\end{enumerate}

Després de realitzar aquests passos ja tenim disponible el projecte al nostre servidor. De totes formes, necessitem que assegurar-nos que aquest funciona de forma continua. Per tal d'aconseguir això s'ha utilitzat la llibreria forever\footnote{\url{https://github.com/nodejitsu/forever/}} que esta disponible a través de Node.js. Per tal de garantir que l'aplicació s'engega i s'apaga de forma automàtica també s'han creat scripts per a que s'executin automàticament cada vegada que es reinicia el servidor.

