

\chapter{Requeriments del sistema}
\label{chap:requeriments}

En aquest capítol es descriuran els requeriments més tècnics del projecte ja que els requeriments funcionals  deriven de les regles  de la botifarra, explicades en el capítol \ref{chap:botifarra}.

Es vol construir un sistema que sigui capàs d'allotjar partides de la botifarra en línia a través d'un navegador. S'ha elegit l'arquitectura en forma de applicació web per tal d'evitar que l'usuari hagi de realitzar la instal·lació de cap tipus de software, facilitant l'accés al sistema. Aquesta decisió implica estructurar el sistema en dos grans blocs: el servidor web i els diferents clients. El primer s'encarrega de proporcionar els mecanismes suficients per comunicar les accions dels usuaris, i els segons s'encarreguen de mostrar la informació que reben del servidor d'una forma amigable.

Aquest sistema també ha de ser capaç d'intercomunicar els diferents jugadors en temps real, fent que les accions d'un jugador es comuniquin de forma quasibé instantànea als altres jugadors. Aquest fet provoca que no es tracti d'una web clàssica, i s'ha de tenir sempre en ment que es tracta d'una aplicació en temps real. Aquest fet condiciona el diseny i la implementació del mateix. 

Així, el sistema ha de proporcionar les següents funcionalitats: 

\begin{itemize}
\item{Permetre la comunicació entre els diferents jugadors connectats al servidor a través d'un xat.}
\item{Allotjar múltiples partides a la vegada.}
\item{Observar partides que estan jugant altres jugadors.}
\item{Crear noves partides.}
\item{Afegir robots per permetre jugar partides sense la necessitat d'altres jugadors.}
\item{
    Notificar als jugadors dels diferents esdeveniments que es van produint durant el transcurs de la partida:
    \begin{itemize}
        \item{S'ha iniciat una partida en que estan involucrats.}
        \item{S'ha iniciat una nova ronda en que estan involucrats.}
        \item{Han d'elegir triomf.}
        \item{Un altre jugador ha elegit triomf per la ronda.}
        \item{Poden elegir si volen contrar la ronda.}
        \item{Un altre jugador ha contrat la ronda.}
        \item{Un altre jugador ha jugat una nova carta.}
        \item{Han de jugar una nova carta.}
        \item{S'ha finalitzat la jugada actual.}
        \item{La ronda en curs s'ha finalitzat.}
        \item{S'ha finalitzat una partida en que estant involucrats.}
    \end{itemize}
}
\end{itemize}

Aquest sistema ha d'utilitzar la tecnologia client-servidor, es a dir hi ha d'haver un programa servidor que es l'encarregat d'allotjar les dades de totes les partides i un programa client que és l'encarregat de mostrar a l'usuari totes aquestes dades. Per garantir que l'experiència del jugador sigui satisfactòria el traspàs d'informació entre el client i el servidor ha de ser en temps real. 

\section{Funcions del servidor}

Com s'ha comentat en l'apartat anterior el programa servidor és el que s'encarrega de proporcionar un canal de comunicació entre els diferents clients. Així, el servidor ha de realitzar les següents tasques:

\begin{enumerate}
	\item{Acceptar connexions per part dels clients.}
	\item{Comunicar-se amb els clients de forma bidireccional, és a dir, ha de ser capaç d'enviar dades als clients  i també rebre'n dades.}
	\item{Allotjar diferents partides de forma simultània.}
	\item{Proporcionar mecanismes necessaris perquè els clients puguin crear partides i veure les que han estat creades per altres usuaris.}
	\item{Notificar als clients de tots els esdeveniments que es produeixen en el joc que estan jugant.}
	\item{Permetre la comunicació entre els diferents clients connectats al servidor}	
\end{enumerate}

\section{Funcions del client}

Els programes client han de realitzar les següents tasques: 

\begin{enumerate}
	\item{Permetre a l'usuari comunicar-se amb el servidor de jocs.}
	\item{Rebre informació del programa servidor i mostrar-la a l'usuari d'una forma amigable.}
	\item{Interactuar amb els usuaris per tal de proporcionar al servidor totes dades que aquests introdueixin.}	
\end{enumerate}



