\chapter{Implementació}

La implementació del joc s'ha realitzar amb dos grans blocs: el servidor i la interfície client. El servidor es l'encarregat de gestionar la informació relativa a les partides disponibles, els jugadors connectats i tota la lògica referent a les partides. La interfície client s'encarrega de interaccionar amb el servidor i mostrar la informació al usuari. 

Per l'intercanvi de dades entre el servidor i el client s'utilitzen WebSockets\footnote{Veure Capítol \ref{sec:websockets}}, ja que ens proporcionen un canal de comunicació bi-direccional a través del protocol HTTP. Aquesta comunicació es totalment asíncrona i està basada en esdeveniments. Així cada vegada que el servidor envia un missatge al client (o a l'invers) es produeix un esdeveniments al destinatari i s'executa el codi associat per al processament d'aquest esdeveniments (en cas que n'hi hagi). 

\section{Servidor}

El servidor s'encarrega de les següents tasques: 

\begin{itemize}
\item{Proporcionar un servidor web encarregat de servir el programa client per a cada connexió.}
\item{Emmagatzemar les partides en curs i jugadors connectats. Proporcionar esdeveniments a través de WebSockets per a que els usuaris puguin interaccionar a través del programa client. }
\item{Aplicar les regles de la botifarra a cada joc en curs i interactuar amb els diferents clients connectats a una partida}
\end{itemize}

En les properes seccions s'explica més profundament com s'ha realitzat la implementació de cadascuna de les tasques. 

\subsection{Servidor Web}

El servidor web és el proces més important de tot el projecte, ja que és l'encarregat de servir les altres parts de tot el projecte. 

Per la implementació del servidor web s'ha utilitzat el paquet express\footnote{\url{http://expressjs.com/}}. Aquest paquet ens permet desenvolupar servidors web a amb poques línies de codi. Així també disposa de codi pre-programat per tal de realitzar tasques comunes a tots el servidors web. Alguns exemples en són:

\begin{itemize}
\item{Registre de peticions.}
\item{Servir fitxers estàtics (imatges,css,JavaScript,etc.).}
\item{Gestió d'errors}
\item{Gestió de galetes (Cookies).}
\end{itemize}

A part de totes aquests funcionalitats, express també s'integra amb diferents llenguatges de plantilles per simplificar l'escriptura del codi HTML. Alguns exemples de motors de plantilles suportats per express són: 

\begin{description}
\item[Haml] {Implementació de Haml\footnote{\url{http://haml-lang.com/}} }
\item[Jade] {Successor de Haml}
\item[EJS] {JavaScript Incrustat}
\item[CoffeeKup] {Plantilles basades amb CoffeeScript}
\item[jQuery Templates for node]
\end{description}

Per la implementació del projecte s'ha utilitzat el motor de plantilles Jade, ja que ens permeten una gran simplificació del codi HTML.

La conjunció de tots els elements comentats anteriorment ens permeten disposar d'un servidor web sobre node.js. Això vol dir que les peticions es processen de forma asíncrona. 

\subsection{Interconnexió Jugadors}

A part de funcionar com a servidor web el servidor també ha d'intercanviar dades amb els clients connectats amb ells. Per això el servidor crear un WebSocket i proporcionar els mecanismes suficients per a que el client s'hi connecti. Així, quan es carrega la primera pàgina de la partida també es carrega el programa client i aquest es connecta al servidor. 

A través de la comunicació amb WebSockets el servidor es poden realitzar les següent accions: 

\begin{description}
\item[login] {Identificar-se amb el servidor.}
\item[create-game] {Crear una nova partida}
\item[join-game] {Entrar a una partida existent.}
\item[watch-game] {Mirar una partida existent.}
\item[add-bot] {Afegir un robot a una partida existent.}
\item[list-games] {Obtenir una llista de totes les partides disponibles al servidor.}
\item[list-players] {Obtenir una llista de tots els jugadors que hi ha al servidor.}
\item[send] {Enviar un missatge a tota la resta d'usuaris del servidor. }
\end{description}

Cada programa client que es vulgui implementar ha d'utilitzar la comunicació amb WebSockets a través d'aquests esdeveniments per interactuar amb l'usuari que vol jugar al servidor. Així la comunicació amb WebSockets permet que es pugin implementar més d'un programa client sense tenir que fer cap tipus de modificació al programa servidor. 

\subsection{Lògica Botifarra}

El servidor també és l'encarregat de controlar tot el flux de treball de la botifarra. A la figura \ref{fig:buti-workflow} es pot veure una representació del flux de treball d'una partida de la botifarra, des de que es crea la partida, fins que aquesta es finalitzada. En aquesta figura s'indica amb color blau les funcions que realitzar el programa servidor i amb color verd les que ha de realitzar el propi jugador que està connectat a la partida. 

\begin{figure}[htbp]
\hspace*{-1.5in}
\centering\includegraphics{img/butifarra_workflow.png}
\caption{Flux de treball de la botifarra}
\label{fig:buti-workflow}
\end{figure} 

\newpage

Com es pot observar a la figura \ref{fig:buti-workflow} el servidor s'encarrega de realitzar les següents tasques referents a la botifarra: 

\begin{itemize}
\item{Començar la partida quan ja hi ha quatre jugadors. }
\item{Repartir de forma aleatòria els jugadors amb dos equips. }
\item{Tirar de forma aleatòria el primer jugador en elegir triomf.}
\item{Repartir les cartes de forma aleatòria entre els diferents jugadors}
\item{Comunicar el triomf elegit a tots els jugadors.}
\item{Comunicar als jugadors si hi ha algun jugador que contrar.}
\item{Determinar el jugador a moure i notificar-lo de que es el seu torn.}
\item{Validar que la carta jugada es la correcta}
\item{Mostrar les cartes de cada jugada.}
\item{Al final de cada ronda mostrar el resultat parcial de la ronda. }
\item{Al final de cada ronda calcular la puntuació de cada equip. En el cas que hi hagi un equip guanyador finalitzar la partida. En el cas que no hi hagi cap jugador guanyador inicial una nova ronda.}
\end{itemize}

Tota la lògica de la botifarra s'ha realitzat a base d'esdeveniments, es a dir, cada vegada que es produeix un acció que requereix la intervenció del servidor aquest llança un esdeveniment que té associada una funció. Aquesta funció es l'encarregada de realitzar tot el processament i llençar el pròxim esdeveniment en cas de que sigui necessari.  


\subsubsection{Aleatorietat}

Un punt molt important per a que els jugadors disfrutin de la seva partida de la botifarra és l'aleatorietat del repartiment de les cartes. Per garantir aquesta aleatorietat s'ha implementar l'algorisme de Knuth Fisher Yates\footnote{\url{http://en.wikipedia.org/wiki/Fisher-Yates_shuffle}} que permet la generació d'una permutació aleatòria sobre un conjunt finit. En el nostre cas el nostre conjunt finit es correspon amb les 48 cartes de la baralla espanyola utilitzades per jugar a la botifarra. 

Donat un conjunt de nombre de 1 a N, aquest algoritme s'aplica de la següent forma: 
\begin{enumerate}
\item{Escriure els nombre de 1 a N.}
\item{Agafar un nombre aleatori k entre 1 i el nombre d'elements pendents de tractar (inclusiu)}
\item{Contant per la part més baixa, treure el número de la posició k que encara no s'ha tractat i escriure'l a un altre lloc}
\item{Repetir el segon pas fins que s'hagin tractat tots els nombres}
\item{El conjunt de nombres que s'han escrit al pas 3 es una permutació aleatòria del conjunt de nombres inicials}
\end{enumerate}

Aquest algorisme te un cost $O(n^2)$ per això s'ha utilitzat la variant de Knuth que permet optimitzar l'algorisme fins a obtenir un cost de $O(n)$. Aquesta variant el que fa es moure els elements pendents de tractar al final de lla llista. Així l'algorisme es queda expressat de la següent forma: 

\begin{verbatim}
 Per ordenar un conjunt de n elements (índexs 0..n-1):
   Per cada i de n - 1 fins a 1 fer:
        j <- nombre aleatori que compleix que  0 <= j <= i
        canviar a[j] i a[i]
\end{verbatim}

A més a més per tal de garantir la complerta aleatorietat i evitar possibles resultats iguals aquest algorisme s'aplica 12 vegades. Amb això s'aconsegueix emular el comportament humà al barrejar la baralla de cartes. 

\section{Client}

El client s'encarrega de realitzar les següents tasques: 

\begin{itemize}
\item{Interaccionar amb el servidor a través de WebSockets.}
\item{Mostrar al usuari les dades que rep del servidor.}
\end{itemize}

Aquestes funcionalitats s'han implementat de forma separada per poder garantir que els canvis a la interfície d'usuari no afectin al la interacció amb el servidor. Aquesta separació ens permet canviar la forma en que es mostren les dades sense tenir que realitzar cap modificació a la lògica del client. Això també ens permet disposar de dos interfícies d'usuari (una per equips d'escriptori i l'altra per a mòbils, per exemple) que utilitzen els mateixos mecanismes per tal d'interactuar amb el servidor. 

En els següents apartats s'explicam com s'han implementat aquestes dos funcionalitats. 

\subsection{Lògica del client}



\subsection{Visualització de la informació}
