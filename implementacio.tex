\chapter{Implementació}

La implementació del joc s'ha realitzar amb dos grans blocs: el servidor i l'interficie client. El servidor es l'encarregat de gestionar la informació realtiva a les partides disponibles, els jugadors conectats i tota la lògica referent a les partides. La interifice client s'encarrega de interaccionar amb el servidor i mostrar la informació al usuari. 

Per l'intercanvi de dades entre el servidor i el client s'utilitzen websockets\footnote{Veure Capítol \ref{sec:websockets}}, ja que ens proporcionen un canal de comunicació bi-direccional a través del protocol HTTP. Aquesta comunicació es totalment asincrona i està basada en events. Així cada vegada que el servidor envía un missatge al client (o a l'inversa) es produeix un event al destinatari i s'executa el codi asociat per al procesament d'aquest events (en cas que n'hi haigi). 

\section{Servidor}

El servidor s'encarrega de les següents tasques: 

\begin{itemize}
\item{Proporcionar un servidor web encarregat de servir el programa client per a cada connexió.}
\item{Emmagatzemar les partides en curs i jugadors connectats. Proporcionar events a través de Websockets per a que els usuaris puguin interaccionar a través del programa client. }
\item{Applicar les regles de la botifarra a cada joc en curs i interactuar amb els diferents clients conectats a una partida}
\end{itemize}

En les properes seccions s'explica més profundament com s'ha realitzat la implementació de cadascuna de les tasques. 

\subsection{Servidor Web}

El servidor web és el proces més important de tot el projecte, ja que és l'encarregat de servir les altres parts de tot el porjecte. 

Per la implementació del servidor web s'ha utiltizat el packet express\footnote{\url{http://expressjs.com/}}. Aquest paquet ens permet desenvolupar servidors web a amb poques línies de codi. Així també disposa de codi pre-programat per tal de realitzar tasques comunes a tots el servidors web. Alguns exemples en són:

\begin{itemize}
\item{Registre de peticions.}
\item{Servir fitxers estàtics (imatges,css,js,etc.).}
\item{Gestió d'errors}
\item{Gestió de galetes (Cookies).}
\end{itemize}

A part de totes aquests funcionalitats, express també s'integra amb diferents llenguatges de plantilles per simplificar l'escriptura del codi HTML. Alguns exemples de motors de plantilles soportats per express són: 

\begin{description}
\item[Haml] {Implementació de Haml\footnote{\url{http://haml-lang.com/}} }
\item[Jade] {Succesor de haml}
\item[EJS] {JavaScript Incrustat}
\item[CoffeeKup] {Plantilles basades amb CoffeeScript}
\item[jQuery Templates for node]
\end{description}

Per la implementació del projecte s'han utilitzat les plantiless Jade, ja que ens permeten una gran simplificació del codi HTML i s'adaptava perfectament a les nostre necessesitats.

\subsection{Interconnexió Jugadors}

\subsection{Lògica Botifarra}

\section{Client}

\subsection{Lògica del client}

\subsection{Visualització de la informació}
