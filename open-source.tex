\chapter{Implicacions d'un projecte de codi obert.}
\label{chap:codi_obert}

Ja que aquest projecte s'ha desenvolupat en un màster lligat amb el programari lliure, tot el codi i la documentació s'ha fet publica seguint la filosofia del programari lliure. Aquest fet provoca que s'hagi de tenir en compte algunes consideracions especials. En aquest apartat s'explica quines han sigut aquestes consideracions i el treball que s'ha dut a terme en aquesta línia. 

\section{Objectius del codi obert}

El principal objectiu del codi obert es posar a la disposició del public el codi font dels programes/documentació que es realitzen, permetent a qualsevol persona la còpia, modificació i redistribució del mateix. Això permet que els programes puguin evolucionar gràcies a la col·laboració de diferents persones i/o institucions.

Un dels principals objectius quan es publica un projecte amb codi obert és aconseguir augmentar el nombre de col·laboradors al projecte, que poden realitzar les següents tasques:

\begin{itemize}
\item{Aportació de noves funcionalitats.}
\item{Detecció i correcció d'errors.}
\item{Millores de documentació.}
\item{Proves del programa abans del llançament d'una nova versió.}
\item{Traduccions a nous idiomes.}
\item{Aportació de noves idees.}
\item{Avaluació de la qualitat del codi.}
\item{Difusió i promoció del projecte.}
\end{itemize}

Evidentment totes aquestes tasques són molt important per a la millora d'un projecte, però el fet de fer públic el codi font del nostre projecte no ens garanteix que trobem col·laboradors que ens puguin ajudar en les mateixes. Així, per tal d'aconseguir que el public estigui interessant amb col·laborar amb el nostre projecte hem de desenvolupar un projecte útil i de qualitat. A més a més, és important fer difusió del projecte perquè sigui conegut per possibles col·laboradors.  

\section{Requeriments d'un projecte de codi obert.}

Per tal de que un projecte sigui de codi obert només cal que la seva llicència garanteixi els drets  d'us, còpia,  modificació i distribució del mateix a l'usuari final. Cal tenir en compte que molta gent creu que els projectes de codi obert tenen l'obligatorietat de ser gratuïts, cosa que no es certa, ja que hi ha projectes de codi obert de pagament. El que ens determina si un projecte es de codi obert o no és la seva llicència. 

\subsection{Elecció de llicència}

Una llicència és un document on és recullen els permisos i obligacions que té una persona sobre algun tipus d'objecte material. Una llicència involucra dos persones: el llicenciador (persona que emet la llicència) i el llicenciat (persona que accepta la llicència). En informàtica el llicenciador és la persona (o conjunt de persones) que han desenvolupat el programa i el llicenciat és l'usuari final del programa. 

Per sort, no es necessari que nosaltres pensem i escrivim una llicència per al nostre projecte, ja que hi ha una gran varietat de llicències ja escrites que es poden adaptar a les necessitats del nostre projecte. A la pàgina web de la iniciativa del codi obert \footnote{\url{http://www.opensource.org/licenses/category}} podem trobar un ampli resum de les mateixes. Per al nostre projecte es van valorar les següents llicències: 

\begin{itemize}
\item{Apache License, 2.0 (Apache-2.0)}
\item{BSD 3-Clause New or Revised license (BSD-3-Clause)}
\item{BSD 3-Clause Simplified or FreeBSD license (BSD-2-Clause)}
\item{GNU General Public License (GPL)}
\item{GNU Library or Lesser General Public License (LGPL)}
\item{MIT license (MIT)}
\item{Mozilla Public License 2.0 (MPL-2.0)}
\item{Common Development and Distribution License (CDDL-1.0)}
\item{Eclipse Public License (EPL-1.0)}
\end{itemize}

De totes aquestes, la que ens va semblar millor va ser la MIT license, per dos motius: simplicitat i llibertats. Així, el la llicència MIT és de les més reduïdes de totes i garanteix la llibertat de qualsevol usuari, ja que els permet fer amb el nostre projecte el que vulguin (fins i tot redistribuir-lo sota una altra llicència), sempre i quan es mantingui una menció als autors originals del projecte.

\subsection{Llicències del programari utilitzat}

Un aspecte important a l'hora d'elegir la llicència del projecte és que aquesta ha de ser compatible amb les llicències que utilitzen altres llibreries utilitzades en el projecte. 

\begin{table}[htbp]\begin{center}\begin{tabular}{|l|c|}
\hline
\textbf{Component} & \textbf{Llicència} \\ \hline
Node.js & MIT \\ \hline
Jade & MIT\\ \hline
Mocha & MIT \\ \hline
jscoverage & GPLv2 \\ \hline
socket.io & MIT \\ \hline
express & MIT  \\ \hline
should.js & MIT \\ \hline
\end{tabular}\end{center}
\caption{Llicències dels components del projecte}\label{t:llicencies-components}
\end{table} 

Com es pot observar a la taula \ref{t:llicencies-components} les llibreries utilitzades tenen les llicències MIT i GPLv2. La nostra llicència es compatible en ja que ambdós llicències ens permeten l'ús del software. En el cas de que s'hagin de realitzar modificacions sobre el programari inicial per adaptar-lo al nostre projecte, ho podrem fer sempre hi quan es mantingui una nota indicant l'autor (en el software sota la llicència MIT). En el cas d'haver de modificar els programes sota llicència GPLv2, hauríem distribuir-lo amb la mateixa llicència. 

\subsection{Eines per als projectes de codi obert}

Per fomentar la col·laboració amb projectes de codi obert es poden utilitzar les següents eines: 

\begin{itemize}
\item{Sistemes de control de versions}
\item{Sistemes de gestió d'errors (bug trackers)}
\item{Llistes de correu per la distribució d'informació.}
\item{Pàgines web per la distribució del projecte.}
\item{Canals de Xat}
\item{Eines per la revisió del codi font del projecte.}
\end{itemize}

En el nostre projecte s'ha utilitzat les següents eines: Sistema de control d'Errors, Sistema de Gestió d'errors, Eines per la revisió del codi font del projecte. Aquestes eines són accessibles a través de la pàgina de Github del projecte\footnote{\url{https://github.com/pokoli/ButiJS/}}. Allí es pot consultar a través del navegador web, tot el codi font del projecte, tots els errors que hi ha pendents de solucionar (i els que ja han estat solucionats), i es pot proposar modificacions al codi font del nostre projecte. 

A mesura que el projecte vagi creixen es valorarà la necessitat de crear noves eines per millorar la comunicació del projecte. 

\section{GitHub}

GitHub és un servei de d'allotjament de codi per a repositoris Git. A part d'allotjar el teu propi repositori ofereix altres serveis com poden ser Gist\footnote{\url{https://gist.github.com/}} (per a compartir petits trossos de codi), una pàgina Wiki per projecte, allotjament de pàgines web i un petit gestió d'incidències molt flexible. A data 16/12/2011 compta amb 1,184,001 persones i allotja 3,448,813 repositoris.

Github ofereix il·limitats repositoris de forma gratuïta sempre que el seu codi sigui públic. A més a més ofereix repositoris privats per un petit preu mensual. 

Aquests fet l'han fet el candidat ideal per allotjar el codi font del projecte. El codi del projecte ha estat publicat des de el primer moment del desenvolupament. A part d'utilitzar Github per allotjar el codi del projecte també s'ha utilitzat el gestor d'incidències per tal d'anotar els errors que s'han anat trobat durant el desenvolupament del projecte. També s'ha utilitzat l'eina de gestió d'incidències per a anotar les noves funcionalitats que es volien desenvolupar per al projecte. Aquestes fets han facilitat que altres desenvolupadors poguessin conèixer l'estat del projecte en cada moment i que poguessin col·laborar en el desenvolupament del mateix. 

Durant el transcurs del projecte s'han generat 48 incidències, de les quals se n'han solucionat 37. A la taula \ref{t:resum-incidencies} es mostra el nombre d'incidències gestionades segons el seu tipus.

\begin{table}[htbp]\begin{center}\begin{tabular}{|l|c|c|c|}
\hline
\textbf{Tipus} & \textbf{Pendents} & \textbf{Tancades} & \textbf{Total} \\ \hline
Error & 4 & 22 & 26 \\ \hline
Nova  funcionalitat & 5 & 3 & 8 \\ \hline
Reestructuració de codi & 1 & 5 & 6 \\ \hline
Altres & 1 & 7 & 8 \\ \hline
Total & 11 & 37 & 48 \\ \hline
\end{tabular}\end{center}
\caption{Resum d'incidències del projecte}\label{t:resum-incidencies}
\end{table} 

