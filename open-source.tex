\chapter{Codi Obert}

Ja que aquest projecte s'ha desenvolupat com en un master lligat amb el programari lliure, tot el codi i la documentació relacionada amb aquest projecte s'ha fet publica seguint la filosofia del programari lliure. Aquest fet provoca que s'haigi de tenir en compte algunes consideracions especials. En aquest apartat s'explica quines han sigut aquestes consideracions i el treball que s'ha dut a terme en aquesta línea. 

\section{Objectius del codi obert}

El principal objectiu del codi obert es posar a la disposició del public el codi font dels programes/documentació que es realitzen, permeten a qualsevol persona la còpia, modificació i redistribució del mateix. Això permet que els programes puguin evolucionar gràcies a la col·laboració de diferents persones i/o institucions.

Un dels principals objectius quan es llibera un projecte amb codi obert es aconseguir augmentar el nombre de col·laboradors al projecte, que poden realitzar les següents tasques:

\begin{itemize}
\item{Aportació de noves funcionalitats.}
\item{Detecció i correcció d'errors.}
\item{Millores de documentació.}
\item{Proves del programa abans del llancament d'una nova versió.}
\item{Traduccions a nous idiomes.}
\item{Aportació de noves idees.}
\item{Evaluació de la qualitat del codi.}
\item{Difusió i promoció del projecte.}
\end{itemize}

Evidentment totes aquestes tasques són molt important per a la millora d'un projecte, però el fet que obrir el codi font del nostre projecte no ens garanteix que trobem col·laboradors que ens puguin ajudar en les mateixes. Així, per tal d'aconseguir que el public estigui interesant amb col·laborar amb el nostre projecte hem de desenvolupar un projecte útil i de qualitat. A més a més, és important fer-n'hi difusió perquè sigui conegut per a possibles col·laboradors.  

\section{Requeriments d'un projecte de codi obert.}

Per tal de que un projecte sigui de codi obert només cal que la seva llicència garanteixi els drets  d'us, còpia,  modificació i distribució del mateix a l'usuari final. Cal tenir en compte que molta gent creu que els projectes de codi obert tenen l'obligatorietat de ser gratuits, cosa que no es certa, ja que hi ha projectes de codi obert de pagament. El que ens determina si un projecte es de codi obert o no és la seva llicència. 

\subsection{Elecció de llicència}

Una llicència és un document on és recullen els permisos que té una persona sobre algun tipus d'objecte material. Una llicència involucra dos persones: el llicenciador (persona que emet la llicència) i el llicenciat (persona que accepta la llicència). En informàtica aquestes el llicenciador és la persona (o conjunt de persones) que han desenvolupat el programa i el llicenciat és l'usuari final del programa. 

Per sort, no es necessari que nosaltres pensem i escribim una llicència per al nostre projecte, ja que hi ha una gran varietat de llicències ja escrites que es poden adaptar a les necesitats del nostre projecte. A la pàgina web de la inciativa del codi obert \footnote{\url{http://www.opensource.org/licenses/category}} podem trobar un ampli resum de les mateixes. Per al nostre projecte es van valorar les següents llicències: 

\begin{itemize}
\item{Apache License, 2.0 (Apache-2.0)}
\item{BSD 3-Clause "New" or "Revised" license (BSD-3-Clause)}
\item{BSD 3-Clause "Simplified" or "FreeBSD" license (BSD-2-Clause)}
\item{GNU General Public License (GPL)}
\item{GNU Library or "Lesser" General Public License (LGPL)}
\item{MIT license (MIT)}
\item{Mozilla Public License 2.0 (MPL-2.0)}
\item{Common Development and Distribution License (CDDL-1.0)}
\item{Eclipse Public License (EPL-1.0)}
\end{itemize}

De totes aquestes, la que ens va semblar millor va ser la MIT license, per dos motius: simplicitat i llibertats. Així, el la llicència MIT és de les més reduïdes de totes i garanteix la llibertat de qualsevol usuari, ja que els permet fer amb el nostre projecte el que vulguin (fins i tot redistribuïr-lo sota una altra llicència), sempre i quan es mantingui una menció als autors originals del projecte. 

\subsection{Eines per als projectes de codi obert}

Per fomentar la col·laboració amb projectes de codi obert es poden utilitzar les següents eines: 

\begin{itemize}
\item{Sistemes de control de versions}
\item{Sistemes de gestió d'errors (bug trackers)}
\item{Llistes de correu per la distribució d'informació.}
\item{Pàgines web per la distribució del projecte.}
\item{Canals de Xat}
\item{Eines per la revisió del codi font del projecte.}
\end{itemize}

En el nostre projecte s'ha utilitzat les següents eines: Sistema de control d'Errors, Sistema de Gestió d'errors, Eines per la revisió del codi font del projecte. Aquestes eines són accesibles a través de la pàgina de Github del projecte\footnote{\url{https://github.com/pokoli/ButiJS/}}. Allí es pot consultar a través del navegador web, tot el codi font del projecte, tots els errors que hi ha pendents de solcuionar (i els que ja han estat solucionats), i es pot proposar modificacions al codi font del nostre projecte. 

A mesura que el projecte vaigi creixen es valorarà la necesitat de crear noves eines per millorar la comunicació del projecte. 

