\chapter{Conclusions}
\label{chap:conclusions}

Gràcies al desenvolupament d'aquest projecte s'ha pogut conèixer les següents tecnologies/metodologies: 

\begin{itemize}
\item{Test Driven Development}
\item{Behavoir Driven Development}
\item{HTML5 en especial l'objecte canvas}
\item{WebSockets}
\item{Node.js}
\end{itemize}

El Test Driven Development i el Behavoir Driven Development són metodologies que permeten augmentar la productivitat en el desenvolupament de l'aplicació, ja que faciliten la comprovació de que el sistema funciona de forma especificada. També cal tenir en compte que la seva utilització augmenta el nombre de línies de codi a escriure durant la fase de codificació, però aquestes línies s'aprofiten amb el fet de que s'utilitzen per comprovar que l'aplicació actua tal hi com es va dissenyar. 

La introducció de l'objecte canvas en la cinquena versió del llenguatge HTML ha suposat un gran canvi en el desenvolupament d'aplicacions web, ja que permet la introducció d'animacions de forma estàndard. Això no era possible en versions anteriors d'aquesta llenguatge i per tant es requeria la utilització de tecnologies addicionals. Amb la introducció d'aquest nou objecte s'estandarditza el desenvolupament d'animacions en l'entorn web, dotant-li noves característiques per a que pugi competir amb altres entorns, com pot ser l'entorn d'escriptori. 

Els WebSockets permeten obtenir informació de forma bidireccional en temps real evitant la sobrecàrrega de dades del protocol HTTP. Això permet que el volum de dades que es poden gestionar sigui més elevat que el que es pot gestionar utilitzant les tecnologies existents. Així, aquesta tecnologia facilita encara més la comunicació en temps real a través de la web entre diferents llocs de la terra. 

Tot i que Node.js sigui un framework molt recent té una gran estabilitat. Aquest fet juntament amb la gran quantitat de mòduls que hi ha publicats per la comunitat, fan que sigui un framework molt útil per al desenvolupament d'aplicacions a la web, especialment per al disseny d'aplicacions que han de gestionar grans quantitats de connexions (o usuaris) simultanis. 

Aquestes tecnologies, i en especial les dos primeres, han de permetre que l'entorn web sigui cada vegada més utilitzat i fent que moltes aplicacions que actualment s'executin des d'un entorn d'escriptori es reescriguin per a poder ser utilitzades directament des de el navegador web. Això ja està passant en algunes aplicacions, però personalment crec que aquesta tendència anirà en augment durant els pròxims anys. 
